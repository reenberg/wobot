\documentclass{beamer}
\usepackage[T1]{fontenc}
\usepackage{textcomp}
\usepackage[utf8x]{inputenc}
\usepackage[danish]{babel}
\usepackage[garamond]{mathdesign}
\usepackage{url}
\usepackage{listings}
\usepackage{graphicx}
\usepackage{soul}

\renewcommand{\ttdefault}{pcr} % bedre typewriter font
\renewcommand{\rmdefault}{ugm} % garamond
\renewcommand{\sfdefault}{phv} % sans-serif font

\title{Fladuino}
\subtitle{Funktionel reaktiv programmering på indlejrede enheder}

\author{Martin Dybdal \and Troels Henriksen \and Jesper Reenberg}

\institute{\textrm{Datalogisk Institut, Københavns Universitet}}
\date{\today}

\mode<presentation>
{
  \usetheme{Frankfurt}
  %\usetheme{Warsaw} 
  \definecolor{uofsgreen}{rgb}{.125,.5,.25}
  \definecolor{natvidgreen}{rgb}{.196,.364,.239}
  \definecolor{kugrey}{rgb}{.4,.4,.4}
  \usecolortheme[named=uofsgreen]{structure}
  \usefonttheme[onlylarge]{structuresmallcapsserif}
  \usefonttheme[onlysmall]{structurebold}
}

\logo{\includegraphics[height=1.5cm]{../diku.png}}

\usenavigationsymbolstemplate{} % fjern navigation

\lstset{language     = Haskell,
        extendedchars= true,
        breaklines   = false,
        tabsize      = 2,
        showstringspaces = false,
        basicstyle   = \small\ttfamily,
        commentstyle = \em,
        inputencoding= utf8
      }

\setcounter{tocdepth}{1}

\begin{document}

\frame{\titlepage}


\section{Introduktion}

\subsection{Behovet for hændelser}

\begin{frame}
\frametitle{Behovet for hændelser}

Fundamentet for et reaktivt system er noget at reagere på.

\begin{block}{Essensen af et reaktivt system}
Hændelse $\rightarrow$ System $\rightarrow$ Respons
\end{block}

Begrebet ``hændelse'' dækker over en ekstern begivenhed, en ændring i
miljøet, eller et signal sendt til systemet.
\\
Det er pålagt at vi konstruerer en abstraktion der gør det praktisk at
arbejde med hændelser.

\end{frame}

\subsection{Strategi}
\begin{frame}
\frametitle{Strategi}

Der er to overordnede implementeringsstrategier:

\begin{description}
\item[Asynkrone hændelser,] hvor vi bruger hardwarefaciliteter
  (interrupts) til at acceptere hændelser, selv hvis vi er midt i en
  beregning (for eksempel, i at beregne en respons til en tidligere
  modtaget hændelse).
\item[Synkrone hændelser,] hvor vi eksplicit checker om der er sket en
  ændring i systemets input der svarer til en hændelse, f.eks. hver
  eneste gang systemet ikke har andet at lave.
\end{description}

Synkrone hændelser kan betyde at vi overser hændelser eller spilder
ressourcer.  Hardware-interrupts gør brug af elektroniske kredsløb der
direkte er designet til at løse dette problem.

\end{frame}

\subsection{Asynkrone hændelser}
\begin{frame}
\frametitle{Asynkrone hændelser}

\begin{block}{}
Vi baserer hændelser på hardware interrupts, specifikt \textit{Pin
  Change Interrupts}.
\end{block}

Dette betyder at hændelser er totalt knyttet til ændring af
inputværdier på Arduino'ens I/O-pins.  

Det er ikke muligt at have hændelser tilknyttet mere obskure
interrupts, f.eks. timer-interrupts, som vi derfor håndterer specielt.
\pause

Vi kræver ikke en 1-til-1 mapping mellem interrupts og hændelser, idet
visse hændelser kan svare til en værdi på blot en af adskillige I/O pins.

\end{frame}

\section{Hændelseskøen}
\subsection{Asynkrone hændelser versus atomiske delgrafer}
\begin{frame}
\frametitle{Asynkrone hændelser versus atomiske delgrafer}

\begin{block}{Problem}
  Asynkrone hændelser kan ankomme på vilkårlige og uforudsigelige
  tidspunkter, men vi garanterer også at atomisk delgrafer er
  atomiske.
\end{block}

\pause

Løsningen er en hændelseskø:

\begin{itemize}
\item Når vi modtager en hændelse bliver den lagt på en hændelseskø.
\item Hver gang vi er blevet færdige med en atomisk delgraf behandler
  vi det ældste element på hændelseskøen.
\end{itemize}

En følge er at vi ikke kan garantere en responstid på vores behandling
af hændelser.  Fladuino giver ingen real-time garantier.

\end{frame}

\subsection{Designovervejelser for hændelseskøen}
\begin{frame}
\frametitle{Designovervejelser for hændelseskøen}

\begin{itemize}
\item Indsættelse skal være (deterministisk) hurtigt, da
  interrupt-handlers skal terminere så hurtigt som muligt.
\item Køen må ikke gro ubegrænset, idet vi kan ende med for lidt fri
  hukommelse til at evaluere de atomiske delgrafer.
\item Det er nemmere at håndtere manglende plads til en ny hændelse i
  køen, end at håndtere at vi løber tør for stakplads under
  evalueringen af en atomisk delgraf.
\end{itemize}

Vi må have en statisk øvre grænse for køens størrelse.

Vi risikerer at gå glip af hændelser hvis køen er fyldt,  men dette
vil sjældent ske.

Under normal brug vil opfyldning af køen kun ske i sjældne øjeblikke,
men kørsel kan fortsætte.  At korrumpere stakken under evalueringen af
en atomisk delgraf vil sandsynligvis være en katastrofal fejl.

\end{frame}

\subsection{Implementering af hændelseskøen}
\begin{frame}
\frametitle{Implementering af hændelseskøen}

\begin{block}{Designparametre}
\begin{itemize}
\item Hurtig indsættelse
\item Statisk (maksimal-)størrelse
\end{itemize}
\end{block}

Den valgte løsning er en cirkulær liste implementeret over et C array
med statisk størrelse.  Element-indsættelse er garanteret til at
terminere inden for et konstant antal instruktioner (hård realtime).

\end{frame}

\section{Definition af hændelser}
\subsection{Målsætning}
\begin{frame}
\frametitle{Målsætning}

Hændelser har \textit{typer} og \textit{payloads}.

\begin{block}{En hændelsesdefinition}
\begin{itemize} 
\item Liste af I/O pins
\item Prædikatfunktion der kaldes når en interrupt opstår på de
  angivne pins
\item Funktion der kaldes til at give hændelsens payload
\end{itemize}
\end{block}

Payloads allokeres dynamisk:

\begin{itemize}
\item Nondeterministisk køretid (men sandsynligvis hurtig nok på
  Arduino)
\item Der bruges ikke mere plads end der rent faktisk er af
  hændelses-payloads i hændelseskøen
\end{itemize}

\end{frame}

\subsection{Eksempel på definition}
\begin{frame}[fragile]
\frametitle{Eksempel på definition}

\tiny
\begin{verbatim}
instance Event PushButtonPressEvent () where
    setupEvent e@(PushButtonPressEvent (d@(PushButton pin))) = 
      do addDevice d
         pv <- statevar d "press_predicate" 
         let v = H.Var (mkName pv)
         addCImport pv [$ty|() -> Bool|] [$cexp|$id:pv|]
         addCFundef [$cedecl|int $id:pv () {
                               return (digitalRead($int:pin) == HIGH);
                             }|]
         return $ mkEvent e Nothing (Just v)
    interruptPins (PushButtonPressEvent (PushButton pin)) = [DPin pin]
\end{verbatim}
\normalsize

\pause

Vi vedkender os at programmør-interface måske er lidt uoptimalt.

\end{frame}

\subsection{Eksempel på brug}
\begin{frame}[fragile]
\frametitle{Eksempel på brug}

\begin{verbatim}
onEvent (PushButtonPressEvent $ PushButton 2) 
        >>> toggle (diode 13)
\end{verbatim}
\normalsize

\end{frame}

\end{document}
