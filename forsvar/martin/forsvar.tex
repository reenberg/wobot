\documentclass{beamer}
\usepackage[T1]{fontenc}
\usepackage{textcomp}
\usepackage[utf8x]{inputenc}
\usepackage[danish]{babel}
\usepackage[garamond]{mathdesign}
\usepackage{url}
\usepackage{listings}
\usepackage{graphicx}
\usepackage{soul}

\renewcommand{\ttdefault}{pcr} % bedre typewriter font
\renewcommand{\rmdefault}{ugm} % garamond
\renewcommand{\sfdefault}{phv} % sans-serif font

\title{Fladuino}
\subtitle{Funktionel reaktiv programmering på indlejrede enheder}

\author{Martin Dybdal \and Troels Henriksen \and Jesper Reenberg}

\institute{\textrm{Datalogisk Institut, Københavns Universitet}}
\date{\today}

\mode<presentation>
{
  \usetheme{Frankfurt}
  %\usetheme{Warsaw} 
  \definecolor{uofsgreen}{rgb}{.125,.5,.25}
  \definecolor{natvidgreen}{rgb}{.196,.364,.239}
  \definecolor{kugrey}{rgb}{.4,.4,.4}
  \usecolortheme[named=uofsgreen]{structure}
  \usefonttheme[onlylarge]{structuresmallcapsserif}
  \usefonttheme[onlysmall]{structurebold}
}

\logo{\includegraphics[height=1.5cm]{diku.png}}

\usenavigationsymbolstemplate{} % fjern navigation

\lstset{language     = Haskell,
        extendedchars= true,
        breaklines   = false,
        tabsize      = 2,
        showstringspaces = false,
        basicstyle   = \small\ttfamily,
        commentstyle = \em,
        inputencoding= utf8
      }

\setcounter{tocdepth}{1}

\begin{document}

\frame{\titlepage}


\section{Introduction}
\subsection{Flask}
\begin{frame}


\end{frame}

Reification

\subsection{What's a monad?}
\begin{frame}[fragile]
  \frametitle{What's a monad?}

  A triplet:
  \begin{itemize}
  \item<1-> A unary type constructor M
  \item<1-> A lifting function \texttt{unitM\footnote{This function is called \texttt{return} in
    Haskell.} :: a -> M a} that lifts a simple
    value into the monad. Creating a \textit{monadic
      value}. 
  \item<1-> A composition function \texttt{bindM\footnote{\texttt{>>=} in Haskell.} :: M a -> (a -> M b) -> M b}
     that applies a monadic function to a monadic value. 
  \end{itemize}
\pause
Obeying three laws (discussed later):
  \begin{enumerate}
  \item<2-> \lstinline{(unitM v) `bindM` f = f v}
  \item<2-> \lstinline{v `bindM` unitM = v}
  \item<2-> 
  \end{enumerate}
\end{frame}


\end{document}
